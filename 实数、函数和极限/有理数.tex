
\subsection{有理数}

在这个标题下的所有内容里的数都代表有理数。 

\subsubsection{有理数域的序}

\begin{theorem}
每一对数 $a$ 与 $b$ 之间必有且仅有下列关系之一:
$$ a = b, a > b, b > a $$
\end{theorem}

\begin{theorem}
由 $a>b$ 与 $b>c$ 可得 $a>c$。
\end{theorem}

\begin{theorem}
若 $a>b$,必有 $c$,使得:
$$ a > c, c > b $$
\end{theorem}

\subsubsection{有理数的加减法}

\begin{theorem}
$a+b=b+a$ 
\end{theorem}

\begin{theorem}
$(a+b)+c=a+(b+c)$ 
\end{theorem}

\begin{theorem}
$a+0=a$ 
\end{theorem}

\begin{theorem}
对任意数 $a$,存在与其对称的 $-a$,使得 $a+(-a)=0$ 
\end{theorem}

\begin{theorem}
由 $a > b$ 可得 $a+c>b+c$ 
\end{theorem}

\begin{quiz}
证明有理数域内,两个有理数的差存在且唯一。
\end{quiz}

\subsubsection{有理数的乘除法}

\begin{theorem}
$ab=ba$ 
\end{theorem}

\begin{theorem}
$(ab)c=a(bc)$ 
\end{theorem}

\begin{theorem}
$a \times 1 = a$ 
\end{theorem}

\begin{theorem}
对于任意 $a\neq 0$ ,存在 $\frac{1}{a}$,使得 $a \cdot \frac{1}{a} = 1$ 
\end{theorem}

\begin{theorem}
$(a+b)c=ac+bc$ 
\end{theorem}

\begin{theorem}
若 $a>b$,且 $c>0$,则有 $ac>bc$ 
\end{theorem}

\subsubsection{阿基米德公理}

\begin{theorem}
对任意 $c>0$,存在自然数 $n>c$ 
\end{theorem}

\subsubsection{有理数的局限}

\begin{quiz}
证明:不存在由既约自然数 $p$、$q$ 构成的有理数 $\frac{p}{q}$,其平方为 $2$。
\end{quiz}

为了证明,试假定其反面:设有分数 $\frac{p}{q}$,其平方等于 $2$。

\begin{equation}
\left( \frac{p}{q}\right)^2 = 2
\end{equation}

整理得 $p^2=2q^2$,故 $p$ 为偶数。
即存在自然数 $r$,使得 $p = 2r$,且 $q$ 为奇数。
但是将 $p = 2r$ 代入式 (1),得 $q^2 = 2r^2$,
由此推得 $q$ 为偶数。
所得的矛盾证明假设不成立,命题得证。
