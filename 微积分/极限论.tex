
\documentclass{ctexart}

\usepackage{amsfonts}
\usepackage{amsmath}
\newtheorem{theorem}{定理}
\begin{document}
\section{极限的定理}

\theorem 当变量趋于有限极限,变量的算术运算式趋于有限极限,值为有限极限的同算术运算。

\theorem 当变量趋于有限极限,变量的序列决定极限的序列,极限的序列不决定变量的序列。

对数列极限,有斯托尔茨定理:

\theorem 设整序变量 $y_n \to +\infty$,且从某一项开始 $y_{n+1} > y_n$,存在极限
$ \lim_{n\to +\infty} \frac{x_{n+1} - x_n}{y_{n+1}-y_n} $,则变量 $\frac{x_n}{y_n}$
存在极限,且
$$
\lim_{n\to +\infty} \frac{x_n}{y_n} = 
\lim_{n\to +\infty} \frac{x_{n+1} - x_n}{y_{n+1}-y_n}
$$

数列极限的收敛原理:

\theorem 整序变量 $x_n$ 有有限极限 $\iff$
$\forall \varepsilon > 0, \exists N \in \mathbb{N} $
使得当 $n,n' > N$ 时,$\lvert x_{n'} - x_n \rvert < \varepsilon$
\end{document}
