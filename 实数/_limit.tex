\documentclass{ctexart}
\usepackage{amsmath}
\begin{document}
\section{数列极限}
设 ${a_n}$ 是给定的数列,如果有一个实数 $a$ 具有下列性质:对任意给定的一个正整数 $\epsilon$,总是存在一个自然数 $N = N(\epsilon)$,使得当 $n > N$ 时,不等式$$
\lvert a_n - a \rvert < \epsilon
$$成立,那么称 $a$ 是数列 ${a_n}$ 的极限,记为:$$
\lim_{n \to \infty} a_n = a
$$\\

收敛数列存在如下性质:\\

【1】唯一;
【2】稳定,改变数列中有限多项的值,不影响数列的收敛性及极限;
【3】极限于常数大小关系趋于稳定;该项性质可发散为
【3.1】收敛数列必有界;
【3.2】当极限大于某个数时,某项后数列的值大于该数;
【3.3】当某项后数列大于等于某个数时,极限大于等于该数。
【4】有限个收敛数列间可以进行算术运算。
【5】极限于极限大小关系趋稳定;该项性质可发散为
【5.1】当极限大于另一个极限时,某项后数列大于另一个数列;
【5.2】当某项后数列大于等于另一个数列时,极限大于等于另一个极限;
【5.2.1】一个重要的推论是数列极限的夹逼定理:
若数列${b_n}$ 和 ${c_n}$都收敛于数 $a$,且对所有充分大的 $n$,有$$
b_n \leq a_n \leq c_n
$$则数列 ${a_n}$ 也收敛,而且极限为 $a$。
\section{函数极限}


\end{document}
